\chapter{Porovnání s ostatními aplikacemi na trhu}
V rámci teoretické části práce jsem se před psaním vlastní "TODO" aplikace rozhodl porovnat "TODO" aplikace, které již byly uvedeny na trh. Průzkum probíhal v týdnu 15.09.2023 - 22.09.2023.
Na trhu se již vyskytuje několik velkých hráčů na trhu takzvaných "TODO"\footnote{Úkolníky} aplikací.
\section{Microsoft TODO}
Tato aplikace je jednou z populárních řešení todo aplikací pro jednotlivce. Je velmi jednoduchá na ovládání, ale nedisponuje systémem skupin, které nevyžadují rozesílání odkazů, nebo kalendářem. Uživatel si může rozdělit úkoly do jednotlivých skupin a nastavit si upozornění na jejich dokončení. Tyto upozornění jsou výhodou toho, že se jedná o .exe a ne o webovou aplikaci, a tak pro posílání upozornění nemusí mít uživatel otevřený prohlížeč. Dále aplikace disponuje určitou formou propojení s Microsoft Office 365.
\section{Trello}
Trello mířeno hlavně na týmová řešení. Je mírně komplexnější než MS TODO, ale disponuje kalendářem. Trello má dále skupiny, respektive tabule, do kterých může majitel přidávat ostatní uživatele. Uživatelé si mohou přiřadit úkoly sami nebo je může přiřadit majitel. Uživatelé si mohou zapnout sledování jednotlivých úkolů a Trello je poté informuje upozorněními na stránce.
\section{Google Keep}
Google Keep je zaměřen na osobní řešení. Uživatel má možnost provádět základní CRUDové\footnote{CRUD - Create Read Update Delete - Tvořit Číst Aktualizovat Mazat} operace na poznámkách a štítcích. Google Keep nemá funkci kalendáře či skupin, ale místo ní můžeme přidávat jako "spolupracovníky" ostatní osoby s Google účtem. I přesto, že nemá aplikace kalendář tak si uživatel může zapnout upozornění na jednotlivé poznámky.