\chapter{Porovnání s ostatními aplikacemi na trhu}
V rámci teoretické části práce jsem se před psaním vlastní "TODO" aplikace rozhodl porovnat "TODO" aplikace, které již byly uvedeny na trh. Průzkum probíhal v týdnu 15.09.2023 - 22.09.2023.
Na trhu se již vyskytuje několik velkých hráčů na trhu takzvaných "TODO"\footnote{Úkolníky} aplikací.
\section{Microsoft TODO}
Tato aplikace je jednou z populárních řešení todo aplikací pro jednotlivce. Je velmi jednoduchá na ovládání, ale nedisponuje systémem skupin, které nevyžadují rozesílání odkazů, nebo kalendářem. Uživatel si může rozdělit úkoly do jednotlivých kategorií a nastavit si upozornění na jejich dokončení. Tyto upozornění jsou výhodou toho, že se jedná o .exe a ne o webovou aplikaci, a tak pro posílání upozornění nemusí mít uživatel otevřený prohlížeč. Dále aplikace disponuje určitou formou propojení s Microsoft Office 365.
\section{Trello}
Trello míří hlavně na týmová řešení. Je mírně komplexnější než MS TODO, ale disponuje kalendářem. Trello má dále skupiny, respektive tabule, do kterých může majitel přidávat ostatní uživatele. Uživatelé si mohou přiřadit úkoly sami nebo je může přiřadit majitel. Uživatelé si mohou zapnout sledování jednotlivých úkolů a Trello je poté informuje upozorněními na stránce.
\section{Google Keep}
Google Keep je zaměřen na osobní řešení. Uživatel má možnost provádět základní CRUDové\footnote{CRUD - Create Read Update Delete - Tvořit Číst Aktualizovat Mazat} operace na poznámkách a štítcích. Google Keep nemá funkci kalendáře či skupin, ale místo ní můžeme přidávat jako "spolupracovníky" ostatní osoby s Google účtem. I přesto, že nemá aplikace kalendář tak si uživatel může zapnout upozornění na jednotlivé poznámky.
\section{Freelo}
Freelo je z porovnávaných aplikací ta nejkomplexnější, ale také jediná, která není pro náročnější uživatele zdarma a uzavírá některé funkce za platební brány. Freelo nabízí 5 tarifů; tarify jsou seřazeny dle jejich měsíční ceny, Freelo nabízí 10\% slevu na každý měsíc, pokud zaplatíme na rok dopředu, ceny jsou uvedeny bez DPH\cite{freelo}.
\begin{enumerate}
	\item Free
	      \begin{itemize}
	      	\item Zdarma
	      	\item Maximálně 3 projekty
	      	\item Maximálně 4 uživatelé, včetně majitele
	      	\item Různé pohledy na úkoly, včetně kalendáře
	      	\item Maximálně 500 MB souborů
	      \end{itemize}
	\item Freelance
	      \begin{itemize}
	      	\item 990 Kč
	      	\item Přidaný pohled na úkoly pomocí "Mind mapy"
	      	\item Stejné funkce, co tarif "Free", nicméně bez omezení
	      \end{itemize}
	\item Team
	      \begin{itemize}
	      	\item 2 190 Kč
	      	\item Role "Správce" - umožňuje delegovat spravování projektu na jiné uživatele než majitele
	      	\item Role "Projekťák" - umožňuje delegovat zakládání projektů na jiné uživatele než majitele
	      	\item Vazby mezi úkoly napříč projekty
	      \end{itemize}
	\item Business
	      \begin{itemize}
	      	\item Cena se odvíjí dle počtu uživatelů
	      	      \begin{itemize}
	      	      	\item Do 5 uživatelů: 1 690 Kč
	      	      	\item Do 10 uživatelů: 2 690 Kč
	      	      	\item Do 15 uživatelů: 3 590 Kč
	      	      	\item Do 20 uživatelů: 4 180 Kč
	      	      	\item Nad 20 uživatelů: 4 180 Kč + 209 Kč za 21. uživatele a dál\footnote{Za každou započatou 20 uživatelů se sníží celkový poplatek za každého uživatele o 16 Kč až do 160 Kč za uživatele}
	      	      \end{itemize}
	      	\item Vše, co předchozí tarify
	      	\item Vlastní role
	      	\item Rozšíření vazeb
	      	\item Přehledy
	      	\item Rozšířené zabezpečení
	      	\item Genttův diagram
	      \end{itemize}
	\item Enterprise
	      \begin{itemize}
	      	\item Od 45 900 Kč
	      	\item Vše, co předchozí tarify
	      	\item Integrace na míru
	      	\item Osobní komunikace s Freelem
	      \end{itemize}
\end{enumerate}